\documentclass[]{article}
\usepackage{lmodern}
\usepackage{amssymb,amsmath}
\usepackage{ifxetex,ifluatex}
\usepackage{fixltx2e} % provides \textsubscript
\ifnum 0\ifxetex 1\fi\ifluatex 1\fi=0 % if pdftex
  \usepackage[T1]{fontenc}
  \usepackage[utf8]{inputenc}
\else % if luatex or xelatex
  \ifxetex
    \usepackage{mathspec}
  \else
    \usepackage{fontspec}
  \fi
  \defaultfontfeatures{Ligatures=TeX,Scale=MatchLowercase}
\fi
% use upquote if available, for straight quotes in verbatim environments
\IfFileExists{upquote.sty}{\usepackage{upquote}}{}
% use microtype if available
\IfFileExists{microtype.sty}{%
\usepackage{microtype}
\UseMicrotypeSet[protrusion]{basicmath} % disable protrusion for tt fonts
}{}
\usepackage[margin=1in]{geometry}
\usepackage{hyperref}
\hypersetup{unicode=true,
            pdftitle={Avaliação de Clusters - Artigo Transando perfil de Personalidade},
            pdfauthor={Leandro Sampaio Silva},
            pdfborder={0 0 0},
            breaklinks=true}
\urlstyle{same}  % don't use monospace font for urls
\usepackage{color}
\usepackage{fancyvrb}
\newcommand{\VerbBar}{|}
\newcommand{\VERB}{\Verb[commandchars=\\\{\}]}
\DefineVerbatimEnvironment{Highlighting}{Verbatim}{commandchars=\\\{\}}
% Add ',fontsize=\small' for more characters per line
\usepackage{framed}
\definecolor{shadecolor}{RGB}{248,248,248}
\newenvironment{Shaded}{\begin{snugshade}}{\end{snugshade}}
\newcommand{\AlertTok}[1]{\textcolor[rgb]{0.94,0.16,0.16}{#1}}
\newcommand{\AnnotationTok}[1]{\textcolor[rgb]{0.56,0.35,0.01}{\textbf{\textit{#1}}}}
\newcommand{\AttributeTok}[1]{\textcolor[rgb]{0.77,0.63,0.00}{#1}}
\newcommand{\BaseNTok}[1]{\textcolor[rgb]{0.00,0.00,0.81}{#1}}
\newcommand{\BuiltInTok}[1]{#1}
\newcommand{\CharTok}[1]{\textcolor[rgb]{0.31,0.60,0.02}{#1}}
\newcommand{\CommentTok}[1]{\textcolor[rgb]{0.56,0.35,0.01}{\textit{#1}}}
\newcommand{\CommentVarTok}[1]{\textcolor[rgb]{0.56,0.35,0.01}{\textbf{\textit{#1}}}}
\newcommand{\ConstantTok}[1]{\textcolor[rgb]{0.00,0.00,0.00}{#1}}
\newcommand{\ControlFlowTok}[1]{\textcolor[rgb]{0.13,0.29,0.53}{\textbf{#1}}}
\newcommand{\DataTypeTok}[1]{\textcolor[rgb]{0.13,0.29,0.53}{#1}}
\newcommand{\DecValTok}[1]{\textcolor[rgb]{0.00,0.00,0.81}{#1}}
\newcommand{\DocumentationTok}[1]{\textcolor[rgb]{0.56,0.35,0.01}{\textbf{\textit{#1}}}}
\newcommand{\ErrorTok}[1]{\textcolor[rgb]{0.64,0.00,0.00}{\textbf{#1}}}
\newcommand{\ExtensionTok}[1]{#1}
\newcommand{\FloatTok}[1]{\textcolor[rgb]{0.00,0.00,0.81}{#1}}
\newcommand{\FunctionTok}[1]{\textcolor[rgb]{0.00,0.00,0.00}{#1}}
\newcommand{\ImportTok}[1]{#1}
\newcommand{\InformationTok}[1]{\textcolor[rgb]{0.56,0.35,0.01}{\textbf{\textit{#1}}}}
\newcommand{\KeywordTok}[1]{\textcolor[rgb]{0.13,0.29,0.53}{\textbf{#1}}}
\newcommand{\NormalTok}[1]{#1}
\newcommand{\OperatorTok}[1]{\textcolor[rgb]{0.81,0.36,0.00}{\textbf{#1}}}
\newcommand{\OtherTok}[1]{\textcolor[rgb]{0.56,0.35,0.01}{#1}}
\newcommand{\PreprocessorTok}[1]{\textcolor[rgb]{0.56,0.35,0.01}{\textit{#1}}}
\newcommand{\RegionMarkerTok}[1]{#1}
\newcommand{\SpecialCharTok}[1]{\textcolor[rgb]{0.00,0.00,0.00}{#1}}
\newcommand{\SpecialStringTok}[1]{\textcolor[rgb]{0.31,0.60,0.02}{#1}}
\newcommand{\StringTok}[1]{\textcolor[rgb]{0.31,0.60,0.02}{#1}}
\newcommand{\VariableTok}[1]{\textcolor[rgb]{0.00,0.00,0.00}{#1}}
\newcommand{\VerbatimStringTok}[1]{\textcolor[rgb]{0.31,0.60,0.02}{#1}}
\newcommand{\WarningTok}[1]{\textcolor[rgb]{0.56,0.35,0.01}{\textbf{\textit{#1}}}}
\usepackage{graphicx,grffile}
\makeatletter
\def\maxwidth{\ifdim\Gin@nat@width>\linewidth\linewidth\else\Gin@nat@width\fi}
\def\maxheight{\ifdim\Gin@nat@height>\textheight\textheight\else\Gin@nat@height\fi}
\makeatother
% Scale images if necessary, so that they will not overflow the page
% margins by default, and it is still possible to overwrite the defaults
% using explicit options in \includegraphics[width, height, ...]{}
\setkeys{Gin}{width=\maxwidth,height=\maxheight,keepaspectratio}
\IfFileExists{parskip.sty}{%
\usepackage{parskip}
}{% else
\setlength{\parindent}{0pt}
\setlength{\parskip}{6pt plus 2pt minus 1pt}
}
\setlength{\emergencystretch}{3em}  % prevent overfull lines
\providecommand{\tightlist}{%
  \setlength{\itemsep}{0pt}\setlength{\parskip}{0pt}}
\setcounter{secnumdepth}{0}
% Redefines (sub)paragraphs to behave more like sections
\ifx\paragraph\undefined\else
\let\oldparagraph\paragraph
\renewcommand{\paragraph}[1]{\oldparagraph{#1}\mbox{}}
\fi
\ifx\subparagraph\undefined\else
\let\oldsubparagraph\subparagraph
\renewcommand{\subparagraph}[1]{\oldsubparagraph{#1}\mbox{}}
\fi

%%% Use protect on footnotes to avoid problems with footnotes in titles
\let\rmarkdownfootnote\footnote%
\def\footnote{\protect\rmarkdownfootnote}

%%% Change title format to be more compact
\usepackage{titling}

% Create subtitle command for use in maketitle
\newcommand{\subtitle}[1]{
  \posttitle{
    \begin{center}\large#1\end{center}
    }
}

\setlength{\droptitle}{-2em}

  \title{Avaliação de Clusters - Artigo Transando perfil de Personalidade}
    \pretitle{\vspace{\droptitle}\centering\huge}
  \posttitle{\par}
    \author{Leandro Sampaio Silva}
    \preauthor{\centering\large\emph}
  \postauthor{\par}
      \predate{\centering\large\emph}
  \postdate{\par}
    \date{09 de março de 2019}


\begin{document}
\maketitle

\hypertarget{resumo}{%
\subsubsection{Resumo}\label{resumo}}

\hypertarget{objetivo}{%
\subsection{Objetivo}\label{objetivo}}

Verificar a possibilidade de se traçar um perfil da personalidade
juntando certos aspectos e grupos (clusters). Através da
\url{https://quantdev.ssri.psu.edu/sites/qdev/files/IntroBasicEFA_2017_1013.html}
Foram utilizados os dados provenientes de um estudo feito atraves de
analise fatorial, presume-se que análise fatorial ira discriminar certos
aspectos das persolidades dos individuos, para meu contento espero
conseguir um resultado similar atraves da analise de clusters onde sera
possivel ter uma visao grafista dos atraves de uma analise exploratoria
que viabilizaria uma visao de como os perfis compartilham
caracteristicas ao mesmo tempo que apresentam caracteristicas proprias.

\hypertarget{resultadoconclusao}{%
\subsection{Resultado/Conclusao}\label{resultadoconclusao}}

Verificou se que apesar dos resultados serem significativos, houve uma
grande diferença entre os resultados obtidos, tendo que as tecnicas são
complementares e nao sobrepostas. Para uma visão clara dos dados e
resutlados, se faz necessária a diminuição do número de variáveis de
forma que elas ainda sejam unicas e explicaveis para então a montagem do
cluster afim de verificar as relações em comum e nao a correlacao em si.
\#\# Palavra-Chave cluster,fatorial,personalidades,estatistica,

\hypertarget{bibligrafia}{%
\subsection{Bibligrafia}\label{bibligrafia}}

\url{https://quantdev.ssri.psu.edu/sites/qdev/files/IntroBasicEFA_2017_1013.html}
\url{https://sillasgonzaga.github.io/2016-06-28-clusterizacaoPaises/}

Definição das funções auxiliares para o modelo

\begin{Shaded}
\begin{Highlighting}[]
\NormalTok{wss <-}\StringTok{ }\ControlFlowTok{function}\NormalTok{(d) \{}
  \KeywordTok{sum}\NormalTok{(}\KeywordTok{scale}\NormalTok{(d, }\DataTypeTok{scale =} \OtherTok{FALSE}\NormalTok{)}\OperatorTok{^}\DecValTok{2}\NormalTok{)}
\NormalTok{\}}
\NormalTok{wrap <-}\StringTok{ }\ControlFlowTok{function}\NormalTok{(i, hc, x) \{}
\NormalTok{  cl <-}\StringTok{ }\KeywordTok{cutree}\NormalTok{(hc, i)}
\NormalTok{  spl <-}\StringTok{ }\KeywordTok{split}\NormalTok{(x, cl)}
\NormalTok{  wss <-}\StringTok{ }\KeywordTok{sum}\NormalTok{(}\KeywordTok{sapply}\NormalTok{(spl, wss))}
\NormalTok{  wss}
\NormalTok{\}}
\NormalTok{elbow_plot<-}\StringTok{ }\ControlFlowTok{function}\NormalTok{(data, dissim,met)\{}
\NormalTok{  cl =}\StringTok{ }\KeywordTok{hclust}\NormalTok{(dissim, }\DataTypeTok{method =}\NormalTok{ met)  }
\NormalTok{  res <-}\StringTok{ }\KeywordTok{sapply}\NormalTok{(}\KeywordTok{seq.int}\NormalTok{(}\DecValTok{1}\NormalTok{, }\KeywordTok{nrow}\NormalTok{(data)), }
\NormalTok{                  wrap, }\DataTypeTok{h =}\NormalTok{cl, }
                  \DataTypeTok{x =}\NormalTok{ data)}
  \KeywordTok{return}\NormalTok{ (res)}
\NormalTok{\}}
\end{Highlighting}
\end{Shaded}

Para resolver o problema dos valores ausentes (os NA), poderia ser
aplicada uma técnica robusta, mas como esta é uma análise simples ou
optei por remover os países que tinham algum dado faltando.

\begin{Shaded}
\begin{Highlighting}[]
\NormalTok{dados_s <-}\StringTok{ }\KeywordTok{na.omit}\NormalTok{(}\KeywordTok{read.csv}\NormalTok{(}\StringTok{'dataset/dados_personalidade_32.csv'}\NormalTok{, }\DataTypeTok{sep =} \StringTok{';'}\NormalTok{, }\DataTypeTok{dec =} \StringTok{','}\NormalTok{)) }\OperatorTok\StringTok{ }\KeywordTok{scale}\NormalTok{()}
\end{Highlighting}
\end{Shaded}

\hypertarget{hc_dist}{%
\subsection{2. hc\_dist}\label{hc_dist}}

Calculo da distancia:

\begin{Shaded}
\begin{Highlighting}[]
\NormalTok{distancia <-}\StringTok{ }\KeywordTok{dist}\NormalTok{(dados_s); }
\end{Highlighting}
\end{Shaded}

\hypertarget{a.-grafico-cotovelo}{%
\subsubsection{A. Grafico cotovelo}\label{a.-grafico-cotovelo}}

Vendo o gráfico:

\begin{Shaded}
\begin{Highlighting}[]
\NormalTok{elbow_data <-}\StringTok{ }\KeywordTok{data.frame}\NormalTok{(}\DataTypeTok{ward =} \KeywordTok{elbow_plot}\NormalTok{(}\DataTypeTok{data=}\NormalTok{dados_s, }
                                           \DataTypeTok{dissim =}\NormalTok{ distancia, }
                                           \DataTypeTok{met =} \StringTok{"ward.D"}\NormalTok{),}
                       \DataTypeTok{upgma =} \KeywordTok{elbow_plot}\NormalTok{(}\DataTypeTok{data=}\NormalTok{dados_s, }
                                          \DataTypeTok{dissim =}\NormalTok{ distancia, }
                                          \DataTypeTok{met =} \StringTok{"median"}\NormalTok{),}
                       \DataTypeTok{upgmc =} \KeywordTok{elbow_plot}\NormalTok{(}\DataTypeTok{data=}\NormalTok{dados_s, }
                                          \DataTypeTok{dissim =}\NormalTok{ distancia, }
                                          \DataTypeTok{met =} \StringTok{"centroid"}\NormalTok{),}
                       \DataTypeTok{k =} \KeywordTok{seq.int}\NormalTok{(}\DecValTok{1}\NormalTok{, }\KeywordTok{nrow}\NormalTok{(dados_s))) }\OperatorTok
\StringTok{  }\KeywordTok{gather}\NormalTok{(}\DataTypeTok{key=}\StringTok{"Metodo"}\NormalTok{,}\DataTypeTok{value=}\StringTok{"WSS"}\NormalTok{, }\OperatorTok{-}\NormalTok{k)}
\NormalTok{elbow_data }\OperatorTok
\StringTok{  }\KeywordTok{ggplot}\NormalTok{(}\KeywordTok{aes}\NormalTok{(}\DataTypeTok{x=}\NormalTok{k,}\DataTypeTok{y=}\NormalTok{WSS, }\DataTypeTok{group=}\NormalTok{Metodo, }\DataTypeTok{color=}\NormalTok{Metodo, }\DataTypeTok{linetype=}\NormalTok{Metodo))}\OperatorTok{+}
\StringTok{  }\KeywordTok{geom_line}\NormalTok{(}\DataTypeTok{size=}\DecValTok{1}\NormalTok{)}\OperatorTok{+}\KeywordTok{theme_bw}\NormalTok{()}
\end{Highlighting}
\end{Shaded}

\includegraphics{note_pokemon_avaliacao_demo_files/figure-latex/unnamed-chunk-4-1.pdf}

\hypertarget{b.-quantos-grupos}{%
\subsubsection{B. Quantos grupos?}\label{b.-quantos-grupos}}

\begin{Shaded}
\begin{Highlighting}[]
\NormalTok{ward <-}\StringTok{ }\KeywordTok{hclust}\NormalTok{(distancia, }\DataTypeTok{method =} \StringTok{"ward.D2"}\NormalTok{)}
\NormalTok{grupos <-}\StringTok{ }\KeywordTok{cutree}\NormalTok{(ward, }\DataTypeTok{k =} \DecValTok{25}\NormalTok{)}
\end{Highlighting}
\end{Shaded}

Pelo pacote \texttt{factoextra}

\begin{Shaded}
\begin{Highlighting}[]
\NormalTok{res <-}\StringTok{ }\NormalTok{factoextra}\OperatorTok{::}\KeywordTok{hcut}\NormalTok{(dados_s, }\DataTypeTok{k =} \DecValTok{25}\NormalTok{)}
\NormalTok{factoextra}\OperatorTok{::}\KeywordTok{fviz_cluster}\NormalTok{(res)}\OperatorTok{+}\KeywordTok{theme_bw}\NormalTok{()}
\end{Highlighting}
\end{Shaded}

\includegraphics{note_pokemon_avaliacao_demo_files/figure-latex/unnamed-chunk-6-1.pdf}

Visualizando dendrograma

\begin{Shaded}
\begin{Highlighting}[]
\KeywordTok{fviz_dend}\NormalTok{(res, }\DataTypeTok{rect =} \OtherTok{TRUE}\NormalTok{)}
\end{Highlighting}
\end{Shaded}

\includegraphics{note_pokemon_avaliacao_demo_files/figure-latex/unnamed-chunk-7-1.pdf}

\hypertarget{c.-silhouette}{%
\subsubsection{C. Silhouette}\label{c.-silhouette}}

\begin{Shaded}
\begin{Highlighting}[]
\NormalTok{sil <-}\StringTok{ }\KeywordTok{silhouette}\NormalTok{(grupos, distancia); }\KeywordTok{plot}\NormalTok{(sil)}
\end{Highlighting}
\end{Shaded}

\includegraphics{note_pokemon_avaliacao_demo_files/figure-latex/unnamed-chunk-8-1.pdf}

Pelo pacote \texttt{factoextra}:

\begin{Shaded}
\begin{Highlighting}[]
\KeywordTok{fviz_silhouette}\NormalTok{(res)}
\end{Highlighting}
\end{Shaded}

\begin{verbatim}
##    cluster size ave.sil.width
## 1        1   12          0.05
## 2        2   22          0.05
## 3        3   16          0.00
## 4        4    7          0.12
## 5        5   17          0.01
## 6        6   14         -0.04
## 7        7   10          0.02
## 8        8   12          0.07
## 9        9    6          0.01
## 10      10   11          0.03
## 11      11   14          0.08
## 12      12   12          0.13
## 13      13   10          0.12
## 14      14    6          0.08
## 15      15    4          0.02
## 16      16   15          0.01
## 17      17    5          0.17
## 18      18    7          0.11
## 19      19    8          0.06
## 20      20    2          0.23
## 21      21   19          0.06
## 22      22    4         -0.09
## 23      23    1          0.00
## 24      24    2          0.17
## 25      25    4          0.18
\end{verbatim}

\includegraphics{note_pokemon_avaliacao_demo_files/figure-latex/unnamed-chunk-9-1.pdf}

\hypertarget{hc_cor}{%
\subsection{3. hc\_cor}\label{hc_cor}}

\begin{itemize}
\tightlist
\item
  Calcular a dissimilaridade Se fizer \texttt{corrplot()} dos dados, ele
  calcula a correla??o entre as vari?veis:
\end{itemize}

\begin{Shaded}
\begin{Highlighting}[]
\KeywordTok{corrplot}\NormalTok{(}\KeywordTok{cor}\NormalTok{(dados_s), }\DataTypeTok{order =} \StringTok{"original"}\NormalTok{, }\DataTypeTok{tl.col=}\StringTok{'black'}\NormalTok{, }\DataTypeTok{tl.cex=}\NormalTok{.}\DecValTok{75}\NormalTok{) }
\end{Highlighting}
\end{Shaded}

\includegraphics{note_pokemon_avaliacao_demo_files/figure-latex/unnamed-chunk-10-1.pdf}

Mas n?o ? essa a nossa inten??o, pois queremos saber a correla??o entre
as observa??es. Para isso precisamos transpor com a fun??o \texttt{t()}:

\begin{Shaded}
\begin{Highlighting}[]
\NormalTok{correla<-}\KeywordTok{cor}\NormalTok{(}\KeywordTok{t}\NormalTok{(dados_s)) }\CommentTok{# correlacao entre observacoes}
\NormalTok{correla[}\DecValTok{1}\OperatorTok{:}\DecValTok{5}\NormalTok{, }\DecValTok{1}\OperatorTok{:}\DecValTok{5}\NormalTok{] }\CommentTok{# 5 primeiras linhas e 5 primeiras colunas}
\end{Highlighting}
\end{Shaded}

\begin{verbatim}
##            1           2          3           4           5
## 1 1.00000000  0.11409437  0.2826975  0.05907968  0.06427981
## 2 0.11409437  1.00000000  0.2113127 -0.02471234 -0.43851464
## 3 0.28269750  0.21131273  1.0000000  0.16034031 -0.24846787
## 4 0.05907968 -0.02471234  0.1603403  1.00000000 -0.31422906
## 5 0.06427981 -0.43851464 -0.2484679 -0.31422906  1.00000000
\end{verbatim}

Como o nosso intuito ? calcular a dissimilaridade, devemos transformar a
similaridade em dissimilarida. Uma das formas ? fazer \(1 - cor\):

\begin{Shaded}
\begin{Highlighting}[]
\NormalTok{dissimil <-}\StringTok{ }\DecValTok{1} \OperatorTok{-}\StringTok{ }\NormalTok{correla }\CommentTok{# dissimilaridade}
\NormalTok{dissimil[}\DecValTok{1}\OperatorTok{:}\DecValTok{5}\NormalTok{,}\DecValTok{1}\OperatorTok{:}\DecValTok{5}\NormalTok{] }\CommentTok{# 5 primeiras linhas e 5 primeiras colunas}
\end{Highlighting}
\end{Shaded}

\begin{verbatim}
##           1         2         3         4         5
## 1 0.0000000 0.8859056 0.7173025 0.9409203 0.9357202
## 2 0.8859056 0.0000000 0.7886873 1.0247123 1.4385146
## 3 0.7173025 0.7886873 0.0000000 0.8396597 1.2484679
## 4 0.9409203 1.0247123 0.8396597 0.0000000 1.3142291
## 5 0.9357202 1.4385146 1.2484679 1.3142291 0.0000000
\end{verbatim}

Como a fun??o \texttt{hclust()} precisa usar uma matriz triangular, usar
\texttt{as.dist()} para transformar a matriz em matriz triangular:

\begin{Shaded}
\begin{Highlighting}[]
\NormalTok{dissimil <-}\StringTok{ }\KeywordTok{as.dist}\NormalTok{(dissimil) }\CommentTok{# transforma em um "triangulo" de distancias}
\end{Highlighting}
\end{Shaded}

\hypertarget{a.-grafico-cotovelo-1}{%
\subsubsection{A. Grafico cotovelo}\label{a.-grafico-cotovelo-1}}

Vendo o gr?fico:

\begin{Shaded}
\begin{Highlighting}[]
\NormalTok{elbow_data <-}\StringTok{ }\KeywordTok{data.frame}\NormalTok{(}\DataTypeTok{ward =} \KeywordTok{elbow_plot}\NormalTok{(}\DataTypeTok{data=}\NormalTok{dados_s, }
                                           \DataTypeTok{dissim =}\NormalTok{ dissimil, }
                                           \DataTypeTok{met =} \StringTok{"ward.D"}\NormalTok{),}
                       \DataTypeTok{upgma =} \KeywordTok{elbow_plot}\NormalTok{(}\DataTypeTok{data=}\NormalTok{dados_s, }
                                          \DataTypeTok{dissim =}\NormalTok{ dissimil, }
                                          \DataTypeTok{met =} \StringTok{"average"}\NormalTok{),}
                       \DataTypeTok{upgmc =} \KeywordTok{elbow_plot}\NormalTok{(}\DataTypeTok{data=}\NormalTok{dados_s, }
                                          \DataTypeTok{dissim =}\NormalTok{ dissimil, }
                                          \DataTypeTok{met =} \StringTok{"centroid"}\NormalTok{),}
                       \DataTypeTok{k =} \KeywordTok{seq.int}\NormalTok{(}\DecValTok{1}\NormalTok{, }\KeywordTok{nrow}\NormalTok{(dados_s))) }\OperatorTok
\StringTok{  }\KeywordTok{gather}\NormalTok{(}\DataTypeTok{key=}\StringTok{"Metodo"}\NormalTok{,}\DataTypeTok{value=}\StringTok{"WSS"}\NormalTok{, }\OperatorTok{-}\NormalTok{k)}

\NormalTok{elbow_data }\OperatorTok\StringTok{ }\KeywordTok{filter}\NormalTok{(k}\OperatorTok{<}\DecValTok{30}\NormalTok{)}\OperatorTok
\StringTok{  }\KeywordTok{ggplot}\NormalTok{(}\KeywordTok{aes}\NormalTok{(}\DataTypeTok{x=}\NormalTok{k,}\DataTypeTok{y=}\NormalTok{WSS, }\DataTypeTok{group=}\NormalTok{Metodo, }\DataTypeTok{color=}\NormalTok{Metodo, }\DataTypeTok{linetype=}\NormalTok{Metodo))}\OperatorTok{+}
\StringTok{  }\KeywordTok{geom_line}\NormalTok{(}\DataTypeTok{size=}\DecValTok{1}\NormalTok{)}\OperatorTok{+}\KeywordTok{theme_bw}\NormalTok{()}
\end{Highlighting}
\end{Shaded}

\includegraphics{note_pokemon_avaliacao_demo_files/figure-latex/unnamed-chunk-14-1.pdf}

\hypertarget{b.-quantos-grupos-1}{%
\subsubsection{B. Quantos grupos?}\label{b.-quantos-grupos-1}}

\begin{Shaded}
\begin{Highlighting}[]
\NormalTok{upgma <-}\StringTok{ }\KeywordTok{hclust}\NormalTok{(dissimil, }\DataTypeTok{method =} \StringTok{"average"}\NormalTok{)}
\KeywordTok{plot}\NormalTok{(upgma)}
\end{Highlighting}
\end{Shaded}

\includegraphics{note_pokemon_avaliacao_demo_files/figure-latex/unnamed-chunk-15-1.pdf}

\hypertarget{c.-silhouette-1}{%
\subsubsection{C. Silhouette}\label{c.-silhouette-1}}

\begin{Shaded}
\begin{Highlighting}[]
\NormalTok{grupos <-}\StringTok{ }\KeywordTok{cutree}\NormalTok{(upgma, }\DataTypeTok{k =} \DecValTok{20}\NormalTok{)}
\NormalTok{sil <-}\StringTok{ }\KeywordTok{silhouette}\NormalTok{(grupos, dissimil)}
\KeywordTok{plot}\NormalTok{(sil)}
\end{Highlighting}
\end{Shaded}

\includegraphics{note_pokemon_avaliacao_demo_files/figure-latex/unnamed-chunk-16-1.pdf}

\hypertarget{k_mean}{%
\subsection{4. k\_mean}\label{k_mean}}

\hypertarget{a.-grafico-cotovelo-2}{%
\subsubsection{A. Grafico cotovelo}\label{a.-grafico-cotovelo-2}}

\begin{Shaded}
\begin{Highlighting}[]
\KeywordTok{fviz_nbclust}\NormalTok{(dados_s, kmeans, }\DataTypeTok{method =} \StringTok{"wss"}\NormalTok{)}
\end{Highlighting}
\end{Shaded}

\includegraphics{note_pokemon_avaliacao_demo_files/figure-latex/unnamed-chunk-17-1.pdf}


\end{document}
