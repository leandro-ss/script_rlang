\documentclass[]{article}
\usepackage{lmodern}
\usepackage{amssymb,amsmath}
\usepackage{ifxetex,ifluatex}
\usepackage{fixltx2e} % provides \textsubscript
\ifnum 0\ifxetex 1\fi\ifluatex 1\fi=0 % if pdftex
  \usepackage[T1]{fontenc}
  \usepackage[utf8]{inputenc}
\else % if luatex or xelatex
  \ifxetex
    \usepackage{mathspec}
  \else
    \usepackage{fontspec}
  \fi
  \defaultfontfeatures{Ligatures=TeX,Scale=MatchLowercase}
\fi
% use upquote if available, for straight quotes in verbatim environments
\IfFileExists{upquote.sty}{\usepackage{upquote}}{}
% use microtype if available
\IfFileExists{microtype.sty}{%
\usepackage{microtype}
\UseMicrotypeSet[protrusion]{basicmath} % disable protrusion for tt fonts
}{}
\usepackage[margin=1in]{geometry}
\usepackage{hyperref}
\hypersetup{unicode=true,
            pdftitle={estatistica\_avaliacao\_conjoint\_analys20190629},
            pdfauthor={Leandro Sampaio},
            pdfborder={0 0 0},
            breaklinks=true}
\urlstyle{same}  % don't use monospace font for urls
\usepackage{color}
\usepackage{fancyvrb}
\newcommand{\VerbBar}{|}
\newcommand{\VERB}{\Verb[commandchars=\\\{\}]}
\DefineVerbatimEnvironment{Highlighting}{Verbatim}{commandchars=\\\{\}}
% Add ',fontsize=\small' for more characters per line
\usepackage{framed}
\definecolor{shadecolor}{RGB}{248,248,248}
\newenvironment{Shaded}{\begin{snugshade}}{\end{snugshade}}
\newcommand{\AlertTok}[1]{\textcolor[rgb]{0.94,0.16,0.16}{#1}}
\newcommand{\AnnotationTok}[1]{\textcolor[rgb]{0.56,0.35,0.01}{\textbf{\textit{#1}}}}
\newcommand{\AttributeTok}[1]{\textcolor[rgb]{0.77,0.63,0.00}{#1}}
\newcommand{\BaseNTok}[1]{\textcolor[rgb]{0.00,0.00,0.81}{#1}}
\newcommand{\BuiltInTok}[1]{#1}
\newcommand{\CharTok}[1]{\textcolor[rgb]{0.31,0.60,0.02}{#1}}
\newcommand{\CommentTok}[1]{\textcolor[rgb]{0.56,0.35,0.01}{\textit{#1}}}
\newcommand{\CommentVarTok}[1]{\textcolor[rgb]{0.56,0.35,0.01}{\textbf{\textit{#1}}}}
\newcommand{\ConstantTok}[1]{\textcolor[rgb]{0.00,0.00,0.00}{#1}}
\newcommand{\ControlFlowTok}[1]{\textcolor[rgb]{0.13,0.29,0.53}{\textbf{#1}}}
\newcommand{\DataTypeTok}[1]{\textcolor[rgb]{0.13,0.29,0.53}{#1}}
\newcommand{\DecValTok}[1]{\textcolor[rgb]{0.00,0.00,0.81}{#1}}
\newcommand{\DocumentationTok}[1]{\textcolor[rgb]{0.56,0.35,0.01}{\textbf{\textit{#1}}}}
\newcommand{\ErrorTok}[1]{\textcolor[rgb]{0.64,0.00,0.00}{\textbf{#1}}}
\newcommand{\ExtensionTok}[1]{#1}
\newcommand{\FloatTok}[1]{\textcolor[rgb]{0.00,0.00,0.81}{#1}}
\newcommand{\FunctionTok}[1]{\textcolor[rgb]{0.00,0.00,0.00}{#1}}
\newcommand{\ImportTok}[1]{#1}
\newcommand{\InformationTok}[1]{\textcolor[rgb]{0.56,0.35,0.01}{\textbf{\textit{#1}}}}
\newcommand{\KeywordTok}[1]{\textcolor[rgb]{0.13,0.29,0.53}{\textbf{#1}}}
\newcommand{\NormalTok}[1]{#1}
\newcommand{\OperatorTok}[1]{\textcolor[rgb]{0.81,0.36,0.00}{\textbf{#1}}}
\newcommand{\OtherTok}[1]{\textcolor[rgb]{0.56,0.35,0.01}{#1}}
\newcommand{\PreprocessorTok}[1]{\textcolor[rgb]{0.56,0.35,0.01}{\textit{#1}}}
\newcommand{\RegionMarkerTok}[1]{#1}
\newcommand{\SpecialCharTok}[1]{\textcolor[rgb]{0.00,0.00,0.00}{#1}}
\newcommand{\SpecialStringTok}[1]{\textcolor[rgb]{0.31,0.60,0.02}{#1}}
\newcommand{\StringTok}[1]{\textcolor[rgb]{0.31,0.60,0.02}{#1}}
\newcommand{\VariableTok}[1]{\textcolor[rgb]{0.00,0.00,0.00}{#1}}
\newcommand{\VerbatimStringTok}[1]{\textcolor[rgb]{0.31,0.60,0.02}{#1}}
\newcommand{\WarningTok}[1]{\textcolor[rgb]{0.56,0.35,0.01}{\textbf{\textit{#1}}}}
\usepackage{graphicx,grffile}
\makeatletter
\def\maxwidth{\ifdim\Gin@nat@width>\linewidth\linewidth\else\Gin@nat@width\fi}
\def\maxheight{\ifdim\Gin@nat@height>\textheight\textheight\else\Gin@nat@height\fi}
\makeatother
% Scale images if necessary, so that they will not overflow the page
% margins by default, and it is still possible to overwrite the defaults
% using explicit options in \includegraphics[width, height, ...]{}
\setkeys{Gin}{width=\maxwidth,height=\maxheight,keepaspectratio}
\IfFileExists{parskip.sty}{%
\usepackage{parskip}
}{% else
\setlength{\parindent}{0pt}
\setlength{\parskip}{6pt plus 2pt minus 1pt}
}
\setlength{\emergencystretch}{3em}  % prevent overfull lines
\providecommand{\tightlist}{%
  \setlength{\itemsep}{0pt}\setlength{\parskip}{0pt}}
\setcounter{secnumdepth}{0}
% Redefines (sub)paragraphs to behave more like sections
\ifx\paragraph\undefined\else
\let\oldparagraph\paragraph
\renewcommand{\paragraph}[1]{\oldparagraph{#1}\mbox{}}
\fi
\ifx\subparagraph\undefined\else
\let\oldsubparagraph\subparagraph
\renewcommand{\subparagraph}[1]{\oldsubparagraph{#1}\mbox{}}
\fi

%%% Use protect on footnotes to avoid problems with footnotes in titles
\let\rmarkdownfootnote\footnote%
\def\footnote{\protect\rmarkdownfootnote}

%%% Change title format to be more compact
\usepackage{titling}

% Create subtitle command for use in maketitle
\newcommand{\subtitle}[1]{
  \posttitle{
    \begin{center}\large#1\end{center}
    }
}

\setlength{\droptitle}{-2em}

  \title{estatistica\_avaliacao\_conjoint\_analys20190629}
    \pretitle{\vspace{\droptitle}\centering\huge}
  \posttitle{\par}
    \author{Leandro Sampaio}
    \preauthor{\centering\large\emph}
  \postauthor{\par}
    \date{}
    \predate{}\postdate{}
  

\begin{document}
\maketitle

\hypertarget{carros-esportivos}{%
\section{Carros Esportivos}\label{carros-esportivos}}

\hypertarget{leitura-dos-dados}{%
\subsection{1.Leitura dos dados}\label{leitura-dos-dados}}

\begin{Shaded}
\begin{Highlighting}[]
\NormalTok{sportscar_long <-}\StringTok{ }\KeywordTok{read.csv}\NormalTok{(}\StringTok{"dataset/sportscar_choice_long.csv"}\NormalTok{)}
\NormalTok{sportscar_long}\OperatorTok{$}\NormalTok{seat<-}\KeywordTok{as.factor}\NormalTok{(sportscar_long}\OperatorTok{$}\NormalTok{seat)}
\NormalTok{sportscar <-}\StringTok{ }\KeywordTok{mlogit.data}\NormalTok{(sportscar_long,}
                       \DataTypeTok{shape =} \StringTok{"long"}\NormalTok{, }
                       \DataTypeTok{alt.var =} \StringTok{"alt"}\NormalTok{,}
                       \DataTypeTok{choice =} \StringTok{"choice"}\NormalTok{) }\OperatorTok\StringTok{ }\KeywordTok{select}\NormalTok{(}\OperatorTok{-}\NormalTok{segment)}

\KeywordTok{table}\NormalTok{ (sportscar}\OperatorTok{$}\NormalTok{seat)}
\end{Highlighting}
\end{Shaded}

\begin{verbatim}
## 
##    2    4    5 
## 2013 2006 1981
\end{verbatim}

\begin{Shaded}
\begin{Highlighting}[]
\CommentTok{#relevel(sportscar , levels = levels(seat)["4","5","2"]) # tentativa para trocar o fator seat de orderm para que mostrasse se o valor eh significativo junto do modelo }
\end{Highlighting}
\end{Shaded}

\hypertarget{construir-um-moledo}{%
\subsection{2.Construir um Moledo}\label{construir-um-moledo}}

\begin{Shaded}
\begin{Highlighting}[]
\NormalTok{model <-}\StringTok{ }\KeywordTok{mlogit}\NormalTok{(choice }\OperatorTok{~}\StringTok{ }\DecValTok{0}\OperatorTok{+}\NormalTok{seat}\OperatorTok{+}\NormalTok{trans}\OperatorTok{+}\NormalTok{convert}\OperatorTok{+}\NormalTok{price, }\DataTypeTok{data=}\NormalTok{sportscar);}\CommentTok{#summary(model)}
\end{Highlighting}
\end{Shaded}

\hypertarget{interpretar-os-coeficientes}{%
\subsection{3.Interpretar os
coeficientes}\label{interpretar-os-coeficientes}}

\begin{Shaded}
\begin{Highlighting}[]
\KeywordTok{str}\NormalTok{(}\StringTok{"Conforme ja conehcido do segmento americano carros automaticos tendem a ser sginificativos para o modelo tendo a preferencia do publico norte americano, outro fator importante e o preco que ajuda nas vendas, tambem existe uma preferencia pelo modelo com 5 lugares, contudo isso comparado ao modelo nulo que no caso e de 2 lugares. Também há uma prefenrecia por modelos converciveis"}\NormalTok{)}
\end{Highlighting}
\end{Shaded}

\begin{verbatim}
##  chr "Conforme ja conehcido do segmento americano carros automaticos tendem a ser sginificativos para o modelo tendo "| __truncated__
\end{verbatim}

\hypertarget{desenhar-um-novo-produto-e-ver-o-share-previsto}{%
\subsection{4.Desenhar um novo produto e ver o share
previsto}\label{desenhar-um-novo-produto-e-ver-o-share-previsto}}

\begin{Shaded}
\begin{Highlighting}[]
\CommentTok{# Prevendo os shares do segmento de carros}
\CommentTok{# Your task is to modify my code to predict shares for the "racer" segment}
\CommentTok{# modify the code below so that the segement is set to "racer" for both alternatives}
\CommentTok{# ajustar modelo usando mlogit() }
\NormalTok{products <-}\StringTok{ }\KeywordTok{data.frame}\NormalTok{(}\DataTypeTok{seat =} \KeywordTok{factor}\NormalTok{(}\KeywordTok{c}\NormalTok{(}\DecValTok{2}\NormalTok{, }\DecValTok{2}\NormalTok{), }\DataTypeTok{levels=}\KeywordTok{c}\NormalTok{(}\DecValTok{2}\NormalTok{,}\DecValTok{4}\NormalTok{,}\DecValTok{5}\NormalTok{)), }
                       \DataTypeTok{trans=} \KeywordTok{factor}\NormalTok{(}\KeywordTok{c}\NormalTok{(}\StringTok{"manual"}\NormalTok{, }\StringTok{"auto"}\NormalTok{), }
                                     \DataTypeTok{levels=}\KeywordTok{c}\NormalTok{(}\StringTok{"auto"}\NormalTok{, }\StringTok{"manual"}\NormalTok{)),}
                       \DataTypeTok{convert=}\KeywordTok{factor}\NormalTok{(}\KeywordTok{c}\NormalTok{(}\StringTok{"no"}\NormalTok{, }\StringTok{"no"}\NormalTok{), }
                                      \DataTypeTok{levels=}\KeywordTok{c}\NormalTok{(}\StringTok{"no"}\NormalTok{, }\StringTok{"yes"}\NormalTok{)), }
                       \DataTypeTok{price =} \KeywordTok{c}\NormalTok{(}\DecValTok{35}\NormalTok{, }\DecValTok{30}\NormalTok{), }
                       \DataTypeTok{segment=}\KeywordTok{factor}\NormalTok{(}\KeywordTok{c}\NormalTok{(}\StringTok{"racer"}\NormalTok{, }\StringTok{"racer"}\NormalTok{), }
                                      \DataTypeTok{levels=}\KeywordTok{c}\NormalTok{(}\StringTok{"basic"}\NormalTok{, }\StringTok{"fun"}\NormalTok{, }\StringTok{"racer"}\NormalTok{)))}
\CommentTok{# predict shares for the "racer" segment}
\KeywordTok{predict_mnl}\NormalTok{(model, products)}
\end{Highlighting}
\end{Shaded}

\begin{verbatim}
##       share seat  trans convert price segment
## 1 0.1023488    2 manual      no    35   racer
## 2 0.8976512    2   auto      no    30   racer
\end{verbatim}

\hypertarget{willingess-to-pay-razao-entre-o-coeficinete-do-fator-e-preco}{%
\subsection{5.Willingess to pay (razao entre o coeficinete do fator e
preco)}\label{willingess-to-pay-razao-entre-o-coeficinete-do-fator-e-preco}}

\begin{Shaded}
\begin{Highlighting}[]
\CommentTok{## Divida os coeficientes pelo}
\CommentTok{# valor negativo (pois as pessoas desejam pagar MENOS) da variavel `price` }
\CommentTok{# (escala de $) = q eh o coeficiente 9}
\NormalTok{coef_price <-}\StringTok{ }\DecValTok{-1}\OperatorTok{*}\KeywordTok{coef}\NormalTok{(model)[}\DecValTok{5}\NormalTok{]}
\KeywordTok{coef}\NormalTok{(model)}\OperatorTok{/}\NormalTok{coef_price}
\end{Highlighting}
\end{Shaded}

\begin{verbatim}
##       seat4       seat5 transmanual  convertyes       price 
##  -0.1016566   2.2262181  -6.3863063   1.0530103  -1.0000000
\end{verbatim}


\end{document}
