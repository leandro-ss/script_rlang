\documentclass[]{article}
\usepackage{lmodern}
\usepackage{amssymb,amsmath}
\usepackage{ifxetex,ifluatex}
\usepackage{fixltx2e} % provides \textsubscript
\ifnum 0\ifxetex 1\fi\ifluatex 1\fi=0 % if pdftex
  \usepackage[T1]{fontenc}
  \usepackage[utf8]{inputenc}
\else % if luatex or xelatex
  \ifxetex
    \usepackage{mathspec}
  \else
    \usepackage{fontspec}
  \fi
  \defaultfontfeatures{Ligatures=TeX,Scale=MatchLowercase}
\fi
% use upquote if available, for straight quotes in verbatim environments
\IfFileExists{upquote.sty}{\usepackage{upquote}}{}
% use microtype if available
\IfFileExists{microtype.sty}{%
\usepackage{microtype}
\UseMicrotypeSet[protrusion]{basicmath} % disable protrusion for tt fonts
}{}
\usepackage[margin=1in]{geometry}
\usepackage{hyperref}
\hypersetup{unicode=true,
            pdftitle={Avaliacaoo Individual - Regressao Linear V},
            pdfauthor={Leandro Sampaio},
            pdfborder={0 0 0},
            breaklinks=true}
\urlstyle{same}  % don't use monospace font for urls
\usepackage{color}
\usepackage{fancyvrb}
\newcommand{\VerbBar}{|}
\newcommand{\VERB}{\Verb[commandchars=\\\{\}]}
\DefineVerbatimEnvironment{Highlighting}{Verbatim}{commandchars=\\\{\}}
% Add ',fontsize=\small' for more characters per line
\usepackage{framed}
\definecolor{shadecolor}{RGB}{248,248,248}
\newenvironment{Shaded}{\begin{snugshade}}{\end{snugshade}}
\newcommand{\KeywordTok}[1]{\textcolor[rgb]{0.13,0.29,0.53}{\textbf{#1}}}
\newcommand{\DataTypeTok}[1]{\textcolor[rgb]{0.13,0.29,0.53}{#1}}
\newcommand{\DecValTok}[1]{\textcolor[rgb]{0.00,0.00,0.81}{#1}}
\newcommand{\BaseNTok}[1]{\textcolor[rgb]{0.00,0.00,0.81}{#1}}
\newcommand{\FloatTok}[1]{\textcolor[rgb]{0.00,0.00,0.81}{#1}}
\newcommand{\ConstantTok}[1]{\textcolor[rgb]{0.00,0.00,0.00}{#1}}
\newcommand{\CharTok}[1]{\textcolor[rgb]{0.31,0.60,0.02}{#1}}
\newcommand{\SpecialCharTok}[1]{\textcolor[rgb]{0.00,0.00,0.00}{#1}}
\newcommand{\StringTok}[1]{\textcolor[rgb]{0.31,0.60,0.02}{#1}}
\newcommand{\VerbatimStringTok}[1]{\textcolor[rgb]{0.31,0.60,0.02}{#1}}
\newcommand{\SpecialStringTok}[1]{\textcolor[rgb]{0.31,0.60,0.02}{#1}}
\newcommand{\ImportTok}[1]{#1}
\newcommand{\CommentTok}[1]{\textcolor[rgb]{0.56,0.35,0.01}{\textit{#1}}}
\newcommand{\DocumentationTok}[1]{\textcolor[rgb]{0.56,0.35,0.01}{\textbf{\textit{#1}}}}
\newcommand{\AnnotationTok}[1]{\textcolor[rgb]{0.56,0.35,0.01}{\textbf{\textit{#1}}}}
\newcommand{\CommentVarTok}[1]{\textcolor[rgb]{0.56,0.35,0.01}{\textbf{\textit{#1}}}}
\newcommand{\OtherTok}[1]{\textcolor[rgb]{0.56,0.35,0.01}{#1}}
\newcommand{\FunctionTok}[1]{\textcolor[rgb]{0.00,0.00,0.00}{#1}}
\newcommand{\VariableTok}[1]{\textcolor[rgb]{0.00,0.00,0.00}{#1}}
\newcommand{\ControlFlowTok}[1]{\textcolor[rgb]{0.13,0.29,0.53}{\textbf{#1}}}
\newcommand{\OperatorTok}[1]{\textcolor[rgb]{0.81,0.36,0.00}{\textbf{#1}}}
\newcommand{\BuiltInTok}[1]{#1}
\newcommand{\ExtensionTok}[1]{#1}
\newcommand{\PreprocessorTok}[1]{\textcolor[rgb]{0.56,0.35,0.01}{\textit{#1}}}
\newcommand{\AttributeTok}[1]{\textcolor[rgb]{0.77,0.63,0.00}{#1}}
\newcommand{\RegionMarkerTok}[1]{#1}
\newcommand{\InformationTok}[1]{\textcolor[rgb]{0.56,0.35,0.01}{\textbf{\textit{#1}}}}
\newcommand{\WarningTok}[1]{\textcolor[rgb]{0.56,0.35,0.01}{\textbf{\textit{#1}}}}
\newcommand{\AlertTok}[1]{\textcolor[rgb]{0.94,0.16,0.16}{#1}}
\newcommand{\ErrorTok}[1]{\textcolor[rgb]{0.64,0.00,0.00}{\textbf{#1}}}
\newcommand{\NormalTok}[1]{#1}
\usepackage{graphicx,grffile}
\makeatletter
\def\maxwidth{\ifdim\Gin@nat@width>\linewidth\linewidth\else\Gin@nat@width\fi}
\def\maxheight{\ifdim\Gin@nat@height>\textheight\textheight\else\Gin@nat@height\fi}
\makeatother
% Scale images if necessary, so that they will not overflow the page
% margins by default, and it is still possible to overwrite the defaults
% using explicit options in \includegraphics[width, height, ...]{}
\setkeys{Gin}{width=\maxwidth,height=\maxheight,keepaspectratio}
\IfFileExists{parskip.sty}{%
\usepackage{parskip}
}{% else
\setlength{\parindent}{0pt}
\setlength{\parskip}{6pt plus 2pt minus 1pt}
}
\setlength{\emergencystretch}{3em}  % prevent overfull lines
\providecommand{\tightlist}{%
  \setlength{\itemsep}{0pt}\setlength{\parskip}{0pt}}
\setcounter{secnumdepth}{0}
% Redefines (sub)paragraphs to behave more like sections
\ifx\paragraph\undefined\else
\let\oldparagraph\paragraph
\renewcommand{\paragraph}[1]{\oldparagraph{#1}\mbox{}}
\fi
\ifx\subparagraph\undefined\else
\let\oldsubparagraph\subparagraph
\renewcommand{\subparagraph}[1]{\oldsubparagraph{#1}\mbox{}}
\fi

%%% Use protect on footnotes to avoid problems with footnotes in titles
\let\rmarkdownfootnote\footnote%
\def\footnote{\protect\rmarkdownfootnote}

%%% Change title format to be more compact
\usepackage{titling}

% Create subtitle command for use in maketitle
\newcommand{\subtitle}[1]{
  \posttitle{
    \begin{center}\large#1\end{center}
    }
}

\setlength{\droptitle}{-2em}

  \title{Avaliacaoo Individual - Regressao Linear V}
    \pretitle{\vspace{\droptitle}\centering\huge}
  \posttitle{\par}
    \author{Leandro Sampaio}
    \preauthor{\centering\large\emph}
  \postauthor{\par}
      \predate{\centering\large\emph}
  \postdate{\par}
    \date{15 de julho de 2018}


\begin{document}
\maketitle

\subsection{QUESTÃO 1}\label{questao-1}

Uma estudante de nutrição visa avaliar a relação entre o consumo de
cálcio e o conhecimento sobre cálcio em estudantes de educação física. A
tabela abaixo sumariza os dados obtidos por ela. Existe uma relação
entre o consumo de cálcio e o conhecimento sobre o cálcio nesses
estudantes de educação física? A que conclusão é possível chegar? Assuma
α = 0,05. Com base nos dados apresentados, é possível estimar a
quantidade diária consumida esperada de cálcio em um estudante que
obtivesse um escore de 50 pontos?

\begin{Shaded}
\begin{Highlighting}[]
\NormalTok{avaliacao =}\StringTok{ }\KeywordTok{read.csv}\NormalTok{(}\StringTok{"dataset/avaliacao_calcio.csv"}\NormalTok{)}
\KeywordTok{colnames}\NormalTok{(avaliacao)[}\DecValTok{1}\NormalTok{] <-}\StringTok{ "no"}
\KeywordTok{colnames}\NormalTok{(avaliacao)[}\DecValTok{2}\NormalTok{] <-}\StringTok{ "score"}
\KeywordTok{colnames}\NormalTok{(avaliacao)[}\DecValTok{3}\NormalTok{] <-}\StringTok{ "calcio"}

\KeywordTok{summary}\NormalTok{(avaliacao)}
\end{Highlighting}
\end{Shaded}

\begin{verbatim}
##        no            score          calcio      
##  Min.   : 1.00   Min.   :10.0   Min.   : 450.0  
##  1st Qu.: 5.75   1st Qu.:22.0   1st Qu.: 725.0  
##  Median :10.50   Median :29.0   Median : 799.0  
##  Mean   :10.50   Mean   :29.6   Mean   : 785.1  
##  3rd Qu.:15.25   3rd Qu.:38.5   3rd Qu.: 895.5  
##  Max.   :20.00   Max.   :48.0   Max.   :1085.0
\end{verbatim}

\begin{verbatim}
## 
## Call:
## lm(formula = score ~ calcio, data = avaliacao)
## 
## Residuals:
##      Min       1Q   Median       3Q      Max 
## -12.3225  -2.6761  -0.6142   1.0614  11.7650 
## 
## Coefficients:
##               Estimate Std. Error t value Pr(>|t|)    
## (Intercept) -14.372996   5.658649  -2.540   0.0205 *  
## calcio        0.056009   0.007044   7.951 2.67e-07 ***
## ---
## Signif. codes:  0 '***' 0.001 '**' 0.01 '*' 0.05 '.' 0.1 ' ' 1
## 
## Residual standard error: 5.355 on 18 degrees of freedom
## Multiple R-squared:  0.7784, Adjusted R-squared:  0.7661 
## F-statistic: 63.22 on 1 and 18 DF,  p-value: 2.675e-07
\end{verbatim}

\begin{verbatim}
## 
##  Shapiro-Wilk normality test
## 
## data:  residuals(mod)
## W = 0.90366, p-value = 0.04833
\end{verbatim}

\includegraphics{note_estatisticaV_regressao_linear_avaliacao_files/figure-latex/unnamed-chunk-2-1.pdf}

\paragraph{Conclusão:}\label{conclusao}

\textbf{R:} \emph{O coeficiente de correlação (\(R^2\)) aponta para
linearidade dos pontos sobre a reta. Tendo o p-valor do coeficiente
angular sido significativo, podemos concluir que a angulação da reta não
é nula, contudo grau de inclinação fica abaixo 1 e apesar de perceptível
a inclinaçao da reta é discreta o que leva a crer que \textbf{existe a
necessidade de uma grande quantidadede calcio para qualquer score (y)
significativo.} }

\subsection{QUESTÃO 2}\label{questao-2}

Visando avaliar o efeito do consumo de álcool (em sujeitos que assumiram
consumir mais de 100 g por dia) sobre a força da contração muscular, um
grupo de pesquisadores (Urbano-Marquez, et al, 1989) avaliou em 50
homens, com auxílio de um miômetro eletrônico, a força de contração do
músculo deltoide do braço não dominante. A tabela ``Álcool.xls'', em
anexo, traz os resultados desses indivíduos expressando uma estimativa
do consumo total de álcool ingerido ao longo da vida (em kg de álcool
por kg de peso corporal) e força muscular (em kgf). Com base nos dados
fornecidos, é possível afirmar a existência de alguma relação entre as
variáveis? Seria possível estabelecer um modelo preditivo a partir dos
dados? Avalie os pré-requisitos necessários. Qual seria a força muscular
esperada do deltoide do braço não dominante de um indivíduo que
apresentasse um consumo de 35 kg de álcool por kg de peso corporal, ao
longo de sua vida?

\begin{Shaded}
\begin{Highlighting}[]
\NormalTok{avaliacao =}\StringTok{ }\KeywordTok{read.csv}\NormalTok{(}\StringTok{"dataset/avaliacao_alcool.csv"}\NormalTok{, }\DataTypeTok{sep =} \StringTok{","}\NormalTok{)}
\KeywordTok{colnames}\NormalTok{(avaliacao)[}\DecValTok{1}\NormalTok{] <-}\StringTok{ "alcool"}
\KeywordTok{colnames}\NormalTok{(avaliacao)[}\DecValTok{2}\NormalTok{] <-}\StringTok{ "musculo"}
\end{Highlighting}
\end{Shaded}

\begin{Shaded}
\begin{Highlighting}[]
\NormalTok{mod <-}\StringTok{ }\KeywordTok{lm}\NormalTok{(alcool }\OperatorTok{~}\StringTok{ }\NormalTok{musculo, avaliacao)}
\end{Highlighting}
\end{Shaded}

\begin{verbatim}
## 
## Call:
## lm(formula = alcool ~ musculo, data = avaliacao)
## 
## Residuals:
##      Min       1Q   Median       3Q      Max 
## -15.9499  -4.7403  -0.0235   4.4356  17.4028 
## 
## Coefficients:
##             Estimate Std. Error t value Pr(>|t|)    
## (Intercept)  49.0316     4.9848   9.836 4.34e-13 ***
## musculo      -1.3915     0.2401  -5.796 5.14e-07 ***
## ---
## Signif. codes:  0 '***' 0.001 '**' 0.01 '*' 0.05 '.' 0.1 ' ' 1
## 
## Residual standard error: 8.401 on 48 degrees of freedom
## Multiple R-squared:  0.4117, Adjusted R-squared:  0.3994 
## F-statistic: 33.59 on 1 and 48 DF,  p-value: 5.136e-07
\end{verbatim}

\begin{verbatim}
## 
##  Shapiro-Wilk normality test
## 
## data:  residuals(mod)
## W = 0.97683, p-value = 0.4273
\end{verbatim}

\includegraphics{note_estatisticaV_regressao_linear_avaliacao_files/figure-latex/unnamed-chunk-5-1.pdf}

\paragraph{Conclusão:}\label{conclusao-1}

\textbf{R:} \emph{O coeficiente de correlação (\(R^2\)) aponta para a
linearidade dos pontos sobre a reta. Tendo o p-valor do coeficiente
angular sido significativo, podemos concluir que a angulação da reta não
é nula, contudo tendo valor estimado para o grau de angulação sido
negativo (-1,40), considera-se que quanto maior a quantidade de alcool
presente no corpo menor será a forca muscular (y).}


\end{document}
